\qns{Fun with Inductors (Challenge)}
\qcontributor{Arda Sahiner}

\begin{enumerate}

\begin{figure}[H]
 \centering
 \begin{circuitikz}
	\draw
	(0, 0)
	to[V=$V_{in}$] ++ (0, -4)
	to[short, -o] ++(6, 0)

	(0, 0)
	to[L, l=$L$, i=$I_L$] ++ (3, 0)
	to[short] ++(2, 0)

	(3, 0)
	to[opening switch,l^=\mbox{$t = 0$}] ++(0, -2)
	to[R, l=$R_1$] ++(0, -2)

	(5, 0)
	to[switch, l^=\mbox{$t = 0$}] ++(0, -2)
	to[R, l=$R_2$] ++ (0, -2)

	(5, -2)
	to[short, -o] ++(1, 0)
	to[open, v^=$V_{out}$] ++(0, -2)

	(3, -1) node[left]{$S_1$}
	(5, -1) node[left]{$S_2$}
	;
\end{circuitikz}
 \caption{Circuit A}
\end{figure}

\qitem Consider circuit A. Assuming that for $t<0$, switch $S_1$ is on and switch $S_2$ is off (and both switches have been in these states indefinitely), what is $i_L(0)$?

\sol{
When $S_1$ is on and $S_2$ is off for a long period of time, $\frac{di_L}{dt} = 0$ because the circuit will have reached a steady state, and the current through $R_1$ will be equal to $i_L$. We find
$$V_{in} - V_L - V_{R_1} = 0$$
$$V_{in} - L\frac{di_L}{dt}(0) - i_L(0)R_1 = 0$$
$$V_{in} - i_L(0)R_1 = 0$$
$$i_L(0) = \frac{V_{in}}{R_1}$$
}

\qitem Now let's assume that for $t \geq 0$, $S_1$ is off and $S_2$ is on. Solve for $V_{out}(t)$ for $t \geq 0$.

\sol {
$$V_{in} - V_L - V_{out} = 0$$
$$V_{in} - L\frac{di_L}{dt} - i_LR_2 = 0$$
$$\frac{di_L}{dt} + \frac{R_2}{L}i_L = \frac{V_{in}}{L}$$

This is a non-homogenous first order differential equation in $i_L$. We can solve for $i_L(t)$ and then use Ohm's law to find $V_{out}(t)$ after this has been solved.
$$\frac{di_L}{dt} + \frac{R_2}{L}(i_L - \frac{V_{in}}{R_2}) = 0$$
Let $\tilde{i_L} = i_L - \frac{V_{in}}{R_2}$. We now have:
$$\frac{d\tilde{i}_L}{dt} + \frac{R_2}{L}\tilde{i}_L = 0$$

The general solution is given by:
$$\tilde{i}_L(t) = c_1 e^{-\frac{R_2}{L}t}$$
Resubstituting back $i_L$, we have:
$$i_L(t) = \frac{V_{in}}{R_2} + c_1 e^{-\frac{R_2}{L}t}$$

Applying initial conditions, we know:
$$i_L(0) = \frac{V_{in}}{R_2} + c_1 = \frac{V_{in}}{R_1}$$
$$c_1 = \frac{V_{in}}{R_1} - \frac{V_{in}}{R_2}$$

Our solution for $i_L(t)$ thus becomes:
$$i_L(t) = \frac{V_{in}}{R_2} + (\frac{V_{in}}{R_1} - \frac{V_{in}}{R_2})e^{-\frac{R_2}{L}t}$$
Since $V_{out}(t) = i_L(t)R_2$, 
$$V_{out}(t) = V_{in}(1 + (\frac{R_2}{R_1} - 1)e^{-\frac{R_2}{L}t})$$
}

\qitem If $V_{in} = 1 V$, $L = 1 nH$, $R_1 = 1 k\Omega$, and $R_2 =10 k\Omega$, what is the maximum value of $V_{out}(t)$ for $t \geq 0$?

\sol{
Since the coefficient in front of our time-varying component $e^{-\frac{R_2}{L}t}$, given by $\frac{R_2}{R_1} - 1 = 9$, is positive, $V_{out}(t)$ undergoes decay over time. Therefore, the maximum value is achieved at $t=0$:
$$\max V_{out} (t) = V_{out}(0) = \frac{R_2}{R_1}V_{in} = 10V$$

}

\qitem In general, if we want $\max{V_{out}(t)}$ to be greater than $V_{in}$, what relationship needs to be maintained between the values of $R_1$ and $R_2$?

\sol{
As long as the coefficient on our exponential term, given by $\frac{R_2}{R_1} - 1$, is greater than $0$ (i.e. when $\frac{R_2}{R_1} > 1$) then the maximum value of $V_{out}(t)$ will be achieved at $t = 0$ and will have a value of $\frac{R_2}{R_1} V_{in} > V_{in}$. Otherwise, if $\frac{R_2}{R_1} \leq 1$, the maximum value of $V_{out}(t)$ is reached at $t=\infty$, where $V_{out} = V_{in}$ regardless of $R_2$ and $R_1$. Therefore, our necessary condition for the maximum of $V_{out}$ to be greater than $V_{in}$ is:
$$R_2 > R_1$$
}

%\begin{figure}[H]
% \centering
% \input{figures/inductor_2}
% \caption{Circuit B}
%\end{figure}

\qitem Now assume that at time $t = t_1$, switch $S_2$ was turned off, and switch $S_1$ was turned back on.  Solve for $i_L(t)$ for $t > t_1$.  If $R_2 > R_1$, how does this $i_L(t)$ for $t>t_1$ compare with the initial condition $i_L(0)$ you found in part (a)?

\sol {
Our new initial condition for $t > t_1$ is given by plugging in $t = t_1$ into the equation for $i_L(t)$ we found in part (b). Thus, $i_L(t_1) = \frac{V_{in}}{R_2} + (\frac{V_{in}}{R_1} - \frac{V_{in}}{R_2})e^{-\frac{R_2}{L}t_1}$. \\
We can write the relationship between the current through the inductor and the current through $R_1$:
$$i_L = i_{R_1}$$
$$i_L = \frac{V_{R_1}}{R_1}$$
$$i_L = \frac{V_{in} - V_L}{R_1}$$
$$i_L = \frac{V_{in}}{R_1} - \frac{L\frac{di_L}{dt}}{R_1}$$
$$\frac{di_L}{dt} + \frac{R_1}{L}i_L = \frac{V_{in}}{L}$$
This is a first order non-homogeneous differential equation similar to that found in part (b), except with $R_1$ in place of $R_2$. Following those steps in part (b), we find the general solution:
$$i_L(t) = \frac{V_{in}}{R_1} + c_1e^{-\frac{R_1}{L}t}$$
To find $c_1$ we apply our initial condition:
$$i_L(t_1) = \frac{V_{in}}{R_1} + c_1e^{-\frac{R_1}{L}t_1} = \frac{V_{in}}{R_2} + (\frac{V_{in}}{R_1} - \frac{V_{in}}{R_2})e^{-\frac{R_2}{L}t_1}$$
$$c_1e^{-\frac{R_1}{L}t_1} = (\frac{V_{in}}{R_2} - \frac{V_{in}}{R_1})(1 - e^{-\frac{R_2}{L}t_1})$$
$$c_1 = \frac{ (\frac{V_{in}}{R_2} - \frac{V_{in}}{R_1})(1 - e^{-\frac{R_2}{L}t_1})}{e^{-\frac{R_1}{L}t_1}}$$
$$c_1 = (\frac{V_{in}}{R_2} - \frac{V_{in}}{R_1})(e^{\frac{R_1}{L}t_1} - e^{\frac{R_1 - R_2}{L}t_1})$$
Thus, we have
$$i_L(t) = \frac{V_{in}}{R_1} + (\frac{V_{in}}{R_2} - \frac{V_{in}}{R_1})(e^{\frac{R_1}{L}t_1} - e^{\frac{R_1 - R_2}{L}t_1})e^{-\frac{R_1}{L}t}$$
for $t>t_1$.
We also see that as $t \to \infty$, $i_L(t)$ for $t > t_1$ becomes:
$$i_L(t=\infty) = \frac{V_{in}}{R_1} + (\frac{V_{in}}{R_2} - \frac{V_{in}}{R_1})(e^{\frac{R_1}{L}t_1} - e^{\frac{R_1 - R_2}{L}t_1})e^{-\frac{R_1}{L}\infty}$$
$$i_L(t=\infty) = \frac{V_{in}}{R_1} + (\frac{V_{in}}{R_2} - \frac{V_{in}}{R_1})(e^{\frac{R_1}{L}t_1} - e^{\frac{R_1 - R_2}{L}t_1})(0)$$
$$i_L(t=\infty) = \frac{V_{in}}{R_1} = i_L(0)$$

Thus, if we turn $S_2$ back off and $S_1$ back on as was described in this part, we will eventually revert back to the initial state from which we started! Specifically, if $R_2 > R_1$, $i_L(t)$ at $t = t_1$ will be less than our initial condition $i_L(0)$, and $i_L(t)$ will rise to $i_L(0)$ over time.
}
\ws{\newpage \ }

\end{enumerate}
