% Author: Taejin Hwang
% Email: taejin@berkeley.edu

\qns{BIBO Stability}

\meta {
  The mini-lecture here establishes the definitions of BIBO stability and why we can look at the solution $x_{p}(t)$ instead of the output $y(t).$ It also explains why we are only looking at the real part of the eigenvalues, since the imaginary part will not contribute to the magnitude of the output.
}

In this question, we will investigate into the definitions of stability for a scalar system modeled by a first order differential equation of the form with the initial condition $x(0) = x_{0}.$

\begin{equation} \label{eq:h}
\ddt{}{t} x(t) = \lambda x(t) + u(t)
\end{equation}

A function $x(t)$ is bounded by a constant $B$ if: $\abs{x(t)} \leq B < \infty$ for all values of $t.$
Therefore we will say that this system is \textbf{BIBO stable} if for every bounded input $u(t),$ the output $y(t)$ is bounded as well.

The output $y(t)$ will be a function of $x(t)$ in the form:
\begin{equation}
y(t) = \alpha x(t) + \beta u(t)
\end{equation}
As $y(t)$ is a linear combination of $x(t),$ and an already assumed to be bounded input $u(t),$ showing that $y(t)$ is bounded is equivalent to showing that $x(t)$ is bounded.

Recall that the particular solution to the differential equation \eqref{eq:h}, was uniquely determined for $t \geq 0$ as:

\begin{equation} \label{eq:p}
x_{p}(t) = x_{0} e^{\lambda t} + \int_{0}^{t} u(\tau) e^{\lambda(t - \tau)} d\tau
\end{equation}

Although $\lambda$ can be complex, we can observe that $|e^{(a + bj)}| = |e^{a}| |e^{bj}| = |e^{a}| |\cos(b) + j \sin(b)| = |e^{a}|.$

The, $\mathfrak{Im}(\lambda)$ does not affect stability, and will correspond to oscillations that are bounded.
Therefore, we will only consider the effects of $\mathfrak{Re}(\lambda)$ which affect the stability of a system,
and for the purposes of this question, assume that $\lambda$ is a real number.

\begin{enumerate}
  \qitem We will start with a bounded input $u(t) = 0.$ Check if $x_{p}(t)$ is bounded for the three following cases:
  \begin{enumerate}[label=(\roman*)]
    \item $\lambda > 0$
    \item $\lambda = 0$
    \item $\lambda < 0$
  \end{enumerate}

  \vspace{100px}

  \sol {
  Remember that in order for a function, $x(t),$ to be bounded, its absolute value must be less than a constant $B$ \textbf{for all} values of $t.$ \vskip 1pt
  For this zero input, the solution to the differential equation for $t \geq 0,$ will be $x_{p}(t) = x_{0} e^{\lambda t}.$
  \begin{enumerate}[label=(\roman*)]
    \item If $\lambda > 0,$ then $x_{p}(t) = x_{0} e^{\lambda t},$ which is a strictly increasing function. As $t \to \infty, x_{p}(t) \to \infty$ which implies that $x_{p}(t)$ is unbounded.
    \item If $\lambda = 0,$ then $x_{p}(t) = x_{0} e^{0 \cdot t} = x_{0}.$ Therefore, $x_{p}(t)$ is bounded by $x_{0}.$
    \item If $\lambda < 0,$ then $x_{p}(t) =  x_{0} e^{\lambda t}$ which is a strictly decreasing function. As $t \to \infty, x_{p}(t) \to 0$ so this function is bounded by its initial condition $x_{0}.$
  \end{enumerate}
  }

  \qitem Based on the conditions from the previous part, can you say anything about the BIBO stability for the three cases?

  \ws{
  \vspace{100px}
  }


  \sol {
    We've shown that for $\lambda > 0,$ a bounded input $u(t) = 0,$ gave an unbounded output.
    Therefore, this system is \textbf{not} BIBO stable for $\lambda > 0.$ \vskip 1pt
    We however, cannot say that the other two are BIBO stable, since we must show that \textbf{every} bounded input gives rise to an bounded output.
  }

  \qitem Now let's consider an input $u(t) = e^{\lambda t},$ can you say anything about the BIBO stability for $\lambda = 0?$

  \ws{
  \vspace{200px}
  }


  \sol {

    If $u(t) = e^{\lambda t} = 1$ then,
    $$x_{p}(t) = x(0) e^{\lambda t} + \int\limits_{0}^{t} u(\tau) e^{\lambda(t - \tau)} d \tau =  x(0) + \int\limits_{0}^{t} e^{0 \cdot (t - \tau)} d \tau = x(0) + \int\limits_{0}^{t} 1 d \tau = x(0) + t$$

    As $t \to \infty, \ x_{p}(t) \to \infty$ which implies that for $\lambda = 0,$ the system is \textbf{not} BIBO stable.
  }

  \qitem How can we show that when $\lambda < 0,$ the system is indeed BIBO stable? `
  You should start by assuming you have a bounded input $u(t)$ such that $\abs{u(t)} \leq B.$ \textit{Hint: $\abs{\int x(t) dt} \leq \int \abs{x(t)}.$}

  \ws{
  \vspace{200px}
  }

  \sol {
    Suppose we have a bounded input $u(t)$ such that $\abs{u(t)} \leq B.$
    We know our solution is:
    $$x_{p}(t) = x(0) e^{\lambda t} + \int\limits_{0}^{t} u(\tau) e^{\lambda(t - \tau)} d \tau.$$

    We can first use the triangle inequality:
    $$\abs{x_{p}(t)} = \abs{x(0) e^{\lambda t} + \int\limits_{0}^{t} u(\tau) e^{\lambda(t - \tau)} d \tau} \leq
    \abs{x(0) e^{\lambda t}} + \abs{\int\limits_{0}^{t} u(\tau) e^{\lambda(t - \tau)}}$$
    We already know that $x(0) e^{\lambda t}$ is bounded by $x(0)$ when $\lambda < 0.$ Therefore
    $$\abs{x_{p}(t)} \leq \abs{x(0)} + \abs{\int\limits_{0}^{t} u(\tau) e^{\lambda(t - \tau)}}$$
    It now remains to show that the integral is bounded as well.
    We start by pulling out the magnitude of $e^{\lambda t}$ out of the absolute value.
    $$\abs{\int\limits_{0}^{t} u(\tau) e^{\lambda(t - \tau)} d \tau} =
    \abs{e^{\lambda t}} \abs{\int\limits_{0}^{t} u(\tau) e^{-\lambda(\tau)} d \tau}$$
    Then we apply the hint,
    $$\abs{e^{\lambda t}} \abs{\int\limits_{0}^{t} u(\tau) e^{-\lambda(\tau)} d \tau}
    \leq \abs{e^{\lambda t}} \int\limits_{0}^{t} \abs{u(\tau) e^{-\lambda \tau}} d \tau$$
    Then we use the fact that $u(t)$ is bounded by $B.$
    $$\abs{e^{\lambda t}} \int\limits_{0}^{t} \abs{u(\tau) e^{-\lambda \tau}} d \tau
    \leq \abs{e^{\lambda t}}  \int\limits_{0}^{t} \abs{B e^{-\lambda \tau} d \tau} =
    B \abs{e^{\lambda t}} \int\limits_{0}^{t} \abs{e^{-\lambda \tau} d \tau}$$
    Since all of our functions are positive for $t \geq 0,$ we can get rid of the absolute values and take the integral from $0$ to $t$:
    \begin{align*}
    \abs{x_{p}(t)} &\leq \abs{x_{0}} + B \abs{e^{\lambda t}} \int\limits_{0}^{t} \abs{e^{-\lambda \tau} d \tau} = \abs{x_{0}} + B e^{\lambda t} \int\limits_{0}^{t} e^{-\lambda \tau} d \tau \\
    &= \abs{x_{0}} + Be^{\lambda t} (-\frac{1}{\lambda} e^{-\lambda \tau})\biggr\rvert_{0}^{t}
    = \abs{x_{0}} - \frac{B}{\lambda}e^{\lambda t} (e^{-\lambda t} - 1) = \abs{x_{0}} +
    \frac{B}{\lambda} (1 - e^{\lambda t})
    \end{align*}
    The quantity $(1 - e^{\lambda t})$ is bounded by $1$ since $e^{\lambda t}$ is between $0$ and $1$ for all values of $t.$ Therefore,
    $$\abs{x_{p}(t)} \leq \abs{x_{0}} + \frac{B}{\lambda}$$
    We have shown that every bounded input $u(t)$ has a bounded output, so the system is BIBO stable for $\lambda < 0.$
  }


\end{enumerate}
