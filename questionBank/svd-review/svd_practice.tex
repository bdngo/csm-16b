% Author: Elena Jia, Taejin Hwang
% Emails: taejin@berkeley.edu

\qns{(Optional) Basic SVD Practice}

This is a review of the individudal steps of finding the Singular Value Decomposition of an $m \times n$ matrix $A.$

The final answer will be of the form $A = U \Sigma V^{T}$ where $U$ is a $m \times m$ orthonormal matrix, $V$ is a $n \times n$ orthonormal matrix, and $\Sigma$ is a $m \times n$ matrix that is a diagonal matrix with $0$s padded on the right or below depending on the dimensions $m$ and $n.$

We give the following procedure to compute the SVD of a $m \times n$ matrix $A.$

\begin{enumerate}[label=(\roman*)]
  \item \textbf{Step 1: Compute the symmetric matrix $A^{T} A$ or $A A^{T}.$} \vskip 1pt
  $A^{T}A$ will be of dimension $n \times n,$ and $AA^{T}$ will be of dimension $m \times m.$ \vskip 1pt
  For a tall, skinny matrix, where $m > n,$ the SVD will be easier to calculate using $A^{T}A$ while for a short, fat matrix, where $m < n,$ the SVD will be easier to calculate using $AA^{T}.$

  \item \textbf{Step 2: Find the eigenvalues and eigenvectors of $A^{T} A$ or $AA^{T}.$} \vskip 1pt
  If $m > n,$ find the eigenvalues ($\lambda_1, \ldots, \lambda_{n}$) and eigenvectors ($\vec{v}_1, \vec{v}_2, \ldots, \vec{v}_{n}$) of $A^TA$. \\
   If $m < n,$ we find the eigenvalues ($\lambda_1, \ldots, \lambda_{m}$) and eigenvectors ($\vec{u}_{1}, \ \vec{u}_{2}, \ldots, \vec{v}_{m}$) of $AA^{T}.$ \\
   By the spectral theorem for real symmetric matrices, these eigenvectors are orthonormal.

  \item \textbf{Step 3: Compute the singular values $\sigma_i = \sqrt{\lambda_i}$ where $\lambda_i$ are the sorted in descending order eigenvalues of $A^TA$ or $AA^{T}.$} \vskip 1pt
  We know these are all non-negative because $(A\vec{v}_i)^T(A\vec{v}_i) = \|A \vec{v}_i\|^2$ and $(A\vec{v}_i)^T(A\vec{v}_i) =\vec{v}_i^T(A^T A)\vec{v}_i = \lambda_i \vec{v}_i^T\vec{v}_i = \lambda_i$. The corresponding normalized eigenvectors $\vec{v}_i$ form the $V$ matrix.

  \item \textbf{Step 4: Find the corresponding vectors of the $U$ or $V$ matrix}  \vskip 1pt
  If $m > n,$ we use the nonzero values of $\sigma_i,$ and $\vec{v}_i,$ to find corresponding vectors of the $U$ matrix, $\vec{u}_i$ by computing $\vec{u}_i = \frac{A\vec{v}_i}{\sigma_i}$. \\
  If $m < n,$ we use the nonzero values of $\sigma_i$ and $\vec{u}_i,$ to find corresponding vectors of the $V$ matrix, $\vec{v}_i$ by computing $\vec{v}_i = \frac{A^{T} \vec{u}_i}{\sigma_i}$

  These are normalized since $\sigma_i = \|A \vec{v}_i\| = \|A^{T} \vec{u}_i \|$ by the argument above, and orthogonal since $(A\vec{v}_i)^T(A\vec{v}_j) = \vec{v}_i^T(A^T A)\vec{v}_j = \lambda_j \vec{v}_i^T\vec{v}_j = 0$ if $i \neq j$, since $V$ is an orthonormal matrix.

  \item \textbf{Step 5 (for finding the full SVD): Use Gram-Schmidt to complete the $U$ or $V$ matrix} \vskip 1pt
  If $m > n$ we can complete the $U$ matrix by finding $\vec{u}_{n + 1}, \ldots, \vec{u}_{m}$ through Gram-Schmidt. \\
  If $m < n$ we will complete the $V$ matrix by finding $\vec{v}_{m + 1}, \ldots, \vec{v}_{n}$ through Gram-Schmidt.

  Alternatively we can solve for $\vec{v}_{i}$ by computing the null-space of $A$ or $\vec{u}_{i}$ by computing the null-space of $A^{T}$ and then performing Gram-Schmidt on the basis for the respective null-space.

\end{enumerate}

Recall the following SVD forms:

\begin{center} \begin{tabular}{|c|c|}
  \hline
  Compact SVD       & $A = U_r \Sigma_r {V_r}^T$ \\ \hline
  Outer Product SVD & $\sum_{i=1}^{r} \sigma_{i} \vec{u}_{i} \vec{v}_{i}^T$ \\ \hline
  Full SVD          & $A = U \Sigma V^T$ \\ \hline
\end{tabular} \end{center}

\newpage
\begin{enumerate}
  \qitem Given the matrix:
  $$A = \begin{bmatrix}
    1 & 1 & 0 \\
    0 & 1 & 1 \\
  \end{bmatrix}$$

  \textbf{Compute the following:}
  \begin{enumerate}
  \item compact SVD
  \item outer product SVD
  \item full SVD
  \end{enumerate}

  \sol {
    \begin{enumerate}[label=(\roman*)]
      \item \textbf{Step 1:}
        The matrix $A$ is a $2 \times 3$ matrix. Therefore, we will compute $AA^{T}.$
        $$AA^{T} = \begin{bmatrix} 2 & 1 \\ 1 & 2 \end{bmatrix}$$
      \item \textbf{Step 2:}
        We then compute the eigenvalues and eigenvectors of $A A^{T}.$
        \begin{align*}
          \text{det}(AA^{T} - \lambda I) &=
          \text{det}\Big(\begin{bmatrix}
          2 - \lambda  & 1\\
          1 & 2 - \lambda \\
          \end{bmatrix} \Big) \\
          &= (2 - \lambda)^{2} - 1 = \lambda^{2} - 4 \lambda + 3 = (\lambda - 3)(\lambda - 1) = 0
        \end{align*}
        The eigenvalues will then be $\lambda = 3, 1.$
        To find the eigenvectors of $AA^{T},$ we look at the null-space of $AA^{T} - \lambda I.$
        $$AA^{T} - 3I = \begin{bmatrix} -1 & 1 \\ 1 & -1 \end{bmatrix}$$
        A basis for the null-space of $AA^{T} - 3I$ is:
        $$\vec{u}_{1} = \begin{bmatrix} \frac{1}{\sqrt{2}} \\ \frac{1}{\sqrt{2}} \end{bmatrix}$$
        $$AA^{T} - I = \begin{bmatrix} 1 & 1 \\ 1 & 1 \end{bmatrix}$$
        A basis for the null-space of $AA^{T} - I$ is:
        $$\vec{u}_{2} = \begin{bmatrix} -\frac{1}{\sqrt{2}} \\ \frac{1}{\sqrt{2}} \end{bmatrix}$$
        Remember that these vectors form the $U$ matrix.
      \item \textbf{Step 3:}
        The singular values are computed by taking the square root of the eigenvalues of $AA^{T}.$
        $$\sigma_{1} = \sqrt{\lambda_{1}} = \sqrt{3}, \ \sigma_{2} = \sqrt{\lambda_{2}} = 1$$
      \item \textbf{Step 4:}
        We now solve for the vectors in the $V$ matrix through the equation $\vec{v}_i = \frac{A^{T} \vec{u}_i}{\sigma_i}$
        $$\vec{v}_{1} = \frac{A^{T} \vec{u}_{1}}{\sqrt{3}} =
        \begin{bmatrix} \frac{1}{\sqrt{6}} \\ \frac{\sqrt{6}}{3} \\ \frac{1}{\sqrt{6}} \end{bmatrix}$$
        $$\vec{v}_{2} = \frac{A^{T} \vec{u}_{2}}{1} =
        \begin{bmatrix} -\frac{1}{\sqrt{2}} \\ 0 \\ \frac{1}{\sqrt{2}} \end{bmatrix}$$
      \item \textbf{Step 5:}
        We still have one last vector $\vec{v}_{3}$ to solve for. To do this, we compute the null-space of $A$ \\
        A basis for the null-space of $A$ can be computed through Gaussian-Elimination as:
        $$\vec{v}_{3} = \begin{bmatrix} 1 \\ -1 \\ 1 \end{bmatrix}.$$
        We however, must normalize $\vec{v}_{3}$ so that $V$ has orthonormal columns. As a result, we see that
        $$\vec{v}_{3} = \begin{bmatrix} \frac{1}{\sqrt{3}} \\ - \frac{1}{\sqrt{3}} \\ \frac{1}{\sqrt{3}} \end{bmatrix}.$$
    \end{enumerate}
    The final result of our full SVD will be:
    \begin{equation}
      A = \begin{bmatrix} \frac{1}{\sqrt{2}} & -\frac{1}{\sqrt{2}} \\ \frac{1}{\sqrt{2}} & \frac{1}{\sqrt{2}} \end{bmatrix}
      \begin{bmatrix} \sqrt{3} & 0 & 0 \\ 0 & 1 & 0 \end{bmatrix}
      \begin{bmatrix} \frac{1}{\sqrt{6}} & \frac{\sqrt{6}}{3} & \frac{1}{\sqrt{6}} \\
      -\frac{1}{\sqrt{2}} & 0 & \frac{1}{\sqrt{2}} \\
      \frac{1}{\sqrt{3}} & - \frac{1}{\sqrt{3}} & \frac{1}{\sqrt{3}} \end{bmatrix}
    \end{equation}

    The compact SVD is:
    \begin{equation}
      A = \begin{bmatrix} \frac{1}{\sqrt{2}} & -\frac{1}{\sqrt{2}} \\ \frac{1}{\sqrt{2}} & \frac{1}{\sqrt{2}} \end{bmatrix}
      \begin{bmatrix} \sqrt{3} & 0 \\ 0 & 1 \end{bmatrix}
      \begin{bmatrix} \frac{1}{\sqrt{6}} & \frac{\sqrt{6}}{3} & \frac{1}{\sqrt{6}} \\
      -\frac{1}{\sqrt{2}} & 0 & \frac{1}{\sqrt{2}} \\ \end{bmatrix}
    \end{equation}

    The outer product SVD is:
    \begin{equation}
      A = \sqrt{3} \begin{bmatrix} \frac{1}{\sqrt{2}} \\ \frac{1}{\sqrt{2}} \end{bmatrix} \begin{bmatrix} \frac{1}{\sqrt{6}} & \frac{\sqrt{6}}{3} & \frac{1}{\sqrt{6}} \end{bmatrix}
        + 1 \begin{bmatrix} -\frac{1}{\sqrt{2}} \\ \frac{1}{\sqrt{2}} \end{bmatrix} \begin{bmatrix} -\frac{1}{\sqrt{2}} & 0 & \frac{1}{\sqrt{2}} \end{bmatrix}
    \end{equation}
  }

  % \newpage
  % \qitem Given the matrix:
  % $$A = \begin{bmatrix}
  %   1 & -1 \\
  %   0 & 1 \\
  %   1 & 0
  % \end{bmatrix}$$.

  % \textbf{Compute the following:}
  % \begin{enumerate}
  % \item compact SVD
  % \item outer product SVD
  % \item full SVD
  % \end{enumerate}

  % \sol {
  %   \begin{enumerate}[label=(\roman*)]
  %     \item \textbf{Step 1:}
  %       The matrix $A$ is a $3 \times 2$ matrix. Therefore, we will compute $A^{T}A.$
  %       $$A^{T}A = \begin{bmatrix} 2 & -1 \\ -1 & 2 \end{bmatrix}$$
  %     \item \textbf{Step 2:}
  %       We then compute the eigenvalues and eigenvectors of $A A^{T}.$
  %       \begin{align*}
  %         \text{det}(AA^{T} - \lambda I) &=
  %         \text{det}\Big(\begin{bmatrix}
  %         2 - \lambda  & -1\\
  %         -1 & 2 - \lambda \\
  %         \end{bmatrix} \Big) \\
  %         &= (2 - \lambda)^{2} - 1 = \lambda^{2} - 4 \lambda + 3 = (\lambda - 3)(\lambda - 1) = 0
  %       \end{align*}
  %       The eigenvalues will then be $\lambda = 3, 1.$
  %       To find the eigenvectors of $AA^{T},$ we look at the null-space of $AA^{T} - \lambda I.$
  %       $$AA^{T} - 3I = \begin{bmatrix} -1 & -1 \\ -1 & -1 \end{bmatrix}$$
  %       A basis for the null-space of $AA^{T} - 3I$ is:
  %       $$\vec{u}_{1} = \begin{bmatrix} \frac{1}{\sqrt{2}} \\ -\frac{1}{\sqrt{2}} \end{bmatrix}$$
  %       $$AA^{T} - I = \begin{bmatrix} 1 & 1 \\ 1 & 1 \end{bmatrix}$$
  %       A basis for the null-space of $AA^{T} - I$ is:
  %       $$\vec{u}_{2} = \begin{bmatrix} \frac{1}{\sqrt{2}} \\ \frac{1}{\sqrt{2}} \end{bmatrix}$$
  %       Remember that these vectors form the $V$ matrix.
  %     \item \textbf{Step 3:}
  %       The singular values are computed by taking the square root of the eigenvalues of $AA^{T}.$
  %       $$\sigma_{1} = \sqrt{\lambda_{1}} = \sqrt{3}, \ \sigma_{2} = \sqrt{\lambda_{2}} = 1$$
  %     \item \textbf{Step 4:}
  %       We now solve for the vectors in the $U$ matrix through the equation $\vec{u}_i = \frac{A \vec{v}_i}{\sigma_i}$
  %       $$\vec{u}_{1} = \frac{A \vec{v}_{1}}{\sqrt{3}} =
  %       \begin{bmatrix} \frac{1}{\sqrt{6}} \\ \frac{\sqrt{6}}{3} \\ \frac{1}{\sqrt{6}} \end{bmatrix}$$
  %       $$\vec{u}_{2} = \frac{A \vec{v}_{2}}{1} =
  %       \begin{bmatrix} -\frac{1}{\sqrt{2}} \\ 0 \\ \frac{1}{\sqrt{2}} \end{bmatrix}$$
  %     \item \textbf{Step 5:}
  %       We still have one last vector $\vec{u}_{3}$ to solve for. To do this, we compute the null-space of $A$ \\
  %       A basis for the null-space of $A$ can be computed through Gaussian-Elimination as:
  %       $$\vec{u}_{3} = \begin{bmatrix} 1 \\ -1 \\ 1 \end{bmatrix}.$$
  %       We however, must normalize $\vec{u}_{3}$ so that $V$ has orthonormal columns. As a result, we see that
  %       $$\vec{u}_{3} = \begin{bmatrix} \frac{1}{\sqrt{3}} \\ - \frac{1}{\sqrt{3}} \\ \frac{1}{\sqrt{3}} \end{bmatrix}.$$
  %   \end{enumerate}
  %   The final result of our SVD will be:
  %   \begin{equation}
  %     A =
  %     \begin{bmatrix}
  %     \frac{1}{\sqrt{6}} & -\frac{1}{\sqrt{2}} & \frac{1}{\sqrt{3}} \\
  %     \frac{\sqrt{6}}{3} & 0 & -\frac{1}{\sqrt{3}} \\
  %     \frac{1}{\sqrt{6}} & \frac{1}{\sqrt{2}} & \frac{1}{\sqrt{3}} \\ \end{bmatrix}
  %     \begin{bmatrix} \sqrt{3} & 0 \\ 0 & 1 \\ 0 & 0 \end{bmatrix}
  %     \begin{bmatrix}
  %     \frac{1}{\sqrt{2}} & \frac{1}{\sqrt{2}} \\
  %     -\frac{1}{\sqrt{2}} & \frac{1}{\sqrt{2}} \end{bmatrix}
  %   \end{equation}

  %   The compact SVD is:
  %   \begin{equation}
  %     A =
  %     \begin{bmatrix}
  %     \frac{1}{\sqrt{6}} & -\frac{1}{\sqrt{2}} \\
  %     \frac{\sqrt{6}}{3} & 0 \\
  %     \frac{1}{\sqrt{6}} & \frac{1}{\sqrt{2}} \\ \end{bmatrix}
  %     \begin{bmatrix} \sqrt{3} & 0 \\ 0 & 1 \end{bmatrix}
  %     \begin{bmatrix}
  %     \frac{1}{\sqrt{2}} & \frac{1}{\sqrt{2}} \\
  %     -\frac{1}{\sqrt{2}} & \frac{1}{\sqrt{2}} \end{bmatrix}
  %   \end{equation}

  %   The outer product SVD is:
  %   \begin{equation}
  %     A = \sqrt{3} \begin{bmatrix} \frac{1}{\sqrt{6}} \\ \frac{\sqrt{6}}{3} \\ \frac{1}{\sqrt{6}} \end{bmatrix} \begin{bmatrix} \frac{1}{\sqrt{2}} & \frac{1}{\sqrt{2}} \end{bmatrix}
  %       + 1 \begin{bmatrix} -\frac{1}{\sqrt{2}} \\ 0 \\ \frac{1}{\sqrt{2}} \end{bmatrix} \begin{bmatrix} -\frac{1}{\sqrt{2}} & \frac{1}{\sqrt{2}} \end{bmatrix}
  %   \end{equation}
  % }
\end{enumerate}
