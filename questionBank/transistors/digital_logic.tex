%Authors: Kevin Zhu and Naomi Sagan

\qns{Digital Logic and NAND gates} \\
In this class, we're often asked to analyze logic gates. We can also look into how we build them. In this problem, we will create a NAND gate out of transistors, and then use NAND gates to construct an OR gate.
\begin{enumerate}
    \item \textit{Constructing a NAND gate}: A digital logic circuit typically consists of a pull-up network of PMOS transistors between $V_{DD}$ and the output, and a pull-down network of NMOS transistors between the output and ground. 
    \begin{enumerate}
        \qitem We want the pull-up network to connect the output to $V_{DD}$ when A NAND B = 1.
        \textbf{Design a pull-up network for the NAND gate.} \\
        \sol{
        \input{\bank/transistors/figures/nand_pullup}
        }

        \ws{\vspace{120px}}

        \qitem We want the pull-down network to connect the output to ground when A NAND B = 0 \\
        When would A NAND B output 0? \textbf{Design a pull-down network for the NAND gate.} \\
        \sol {
        $$\text{A NAND B = 0 when A=1 and B = 1}$$
        \input{\bank/transistors/figures/nand_pulldown}
        }

        \ws{\vspace{120px}}

        \qitem \textbf{Now put the two networks together to create a NAND gate}. \\
        \sol{\input{\bank/transistors/figures/nand}}

        \ws{\vspace{180px}}

    \end{enumerate}
    Fun fact: NAND gates are universal, which means that every boolean operation can be represented with a combination of NAND gates. Knowing this, let us create an OR gate from NAND gates. \\

    \textbf{Note: These parts display how logic gates can be used but we will not focus on designing these logic operators in this class (this is a bit out of scope so don’t focus about studying this).}

    \qitem First, let's create a intermediate tool to help us build our OR gate. \textbf{How can we build an inverter using just NAND gates?} \\
    \ws{\vspace{90px}}

    \sol{
    $$\text{NOT (A) = A NAND A}$$
    \input{\bank/transistors/figures/NAND_inverter}
    When A is 0, A NAND A = 1. Likewise when A =1, A NAND A = 0.
    }
    \qitem Now, let's examine an OR gate. \textbf{How can you make an OR gate using NOT and NAND gates?}
    
    \emph{Hint: draw out a truth table if you get stuck.} \\

    \ws{\vspace{90px}}

    \sol{$$\text{A OR B = (NOT A) NAND (NOT B)}$$
	\input{\bank/transistors/figures/OR}}
    \qitem \textbf{Finally, replace the NOT gates with their NAND equivalents to make an OR gate solely out of NAND gates.} \\

    \ws{\vspace{90px}}

    \sol{$$\text{A or B = (NOT A) NAND (NOT B) = (A NAND A) NAND (B NAND B)}$$
    See the solution to part (c) for the logic gate diagram!
    }

    \qitem \textbf{OPTIONAL CHALLENGE: Now try making an XOR gate using only NAND gates.} \\

    \ws{\vspace{90px}}


    \sol{You can rewrite A XOR B as (A OR B) AND (A NAND B). Additionally, we can rewrite an AND gate as A AND B = NOT(A NAND B). Thus, we can plug in our gates that we have already created to get:
    \input{\bank/transistors/figures/XOR}}
\end{enumerate}
