\qns{RLC circuit Phasor Analysis}
\qcontributor{Varun Mishra}

In this question, we will take a look at an electrical systems described by second order differential equations and analyze it using the phasor domain. Consider the circuit below where $R=3 \text{k}\Omega$, $L=1\mbox{mH}$, $C=100$nF, and $V_{\text{s}}=5\cos(1000t+\frac{\pi}{4})$:

\begin{center}
		\begin{circuitikz}[scale=0.8]
			\draw (0,4)
			to [sV, l= $V_s$] (0,0)
			(0,4)
			to [R = $R$,v=$V_R$] (4,4)
			to [L = $L$,v=$V_L$,i=$i(t)$] (8,4)
			to [short] (10,4)
			to [C = $C$,v=$V_{\text{out}}$] (10,0)
			to [short] (0,0);
		\end{circuitikz}
	\end{center}
\begin{enumerate}

\qitem What are the impedances of the resistor, inductor and capacitor, $Z_R$, $Z_L$, and $Z_C$?

\ws{
\vspace{75px}
}

\sol{
\begin{align*}
\intertext{The impedance of a resistor is the same as its resistance}
Z_R &= 3000\Omega
\intertext{We can find the frequency of the circuit by looking at $V_{\text{s}}$. The form of a cosine function is $A\cos(\omega t+\phi )$ where $A$ is amplitude, $\omega$ is frequency, and $\phi$ is phase. In this case, the frequency is $1000 \frac{\text{rad}}{\text{s}}$}
Z_L &= j\omega L = j1000*10^{-3}= j1 \Omega \\
Z_C &= \frac{1}{j\omega C} = \frac{1}{j1000*10^{-7}} =-j10^4 \Omega
\end{align*}

}
\qitem Solve for $\widetilde{V}_{\text{out}}$ in phasor form.

\ws{
\vspace{100px}
}

\sol{
\begin{align*}
\intertext{converting $V_{\text{s}}$ into phasor form, we have}
\widetilde{V}_{\text{s}} &= \frac{1}{2} |A|e^{j\phi} = \frac{5}{2}e^{j\frac{\pi}{4}}
\intertext{The circuit given is a voltage divider. Since impedances act like resistors, we can use the same equation as the resistive voltage divider.}
\widetilde{V}_{\text{out}} = \widetilde{V}_{\text{s}}\frac{Z_C}{Z_R+Z_L+Z_C}&= \widetilde{V}_{\text{s}}*\big|\frac{Z_C}{Z_R+Z_L+Z_C}\big|e^{j*\angle\big(\frac{Z_C}{Z_R+Z_L+Z_C}\big)} \tag{1} \\
\intertext{We can solve for the magnitude and angle of the divider using}
\big|\frac{Z_C}{Z_R+Z_L+Z_C}\big|&=\frac{|Z_C|}{|Z_R+Z_L+Z_C|}=\frac{10^4}{\sqrt{3000^2+(1-10^4)^2}} = 0.958 \\
\angle\big(\frac{Z_C}{Z_R+Z_L+Z_C}\big)&= \angle(Z_C)-\angle(Z_R+Z_L+Z_C)=\text{atan2}\Big(\frac{\mathfrak{Im}(Z_C)}{\mathfrak{Re}(Z_C)}\Big)-\text{atan2}\Big(\frac{\mathfrak{Im}(Z_R+Z_L+Z_C)}{\mathfrak{Re}(Z_R+Z_L+Z_C)}\Big) \\
\angle\big(\frac{Z_C}{Z_R+Z_L+Z_C}\big)&=-0.2915 \text{ rad}
\intertext{Plugging back into (1)}
\widetilde{V}_{\text{out}}&=\frac{5}{2}e^{j\frac{\pi}{4}}*0.958e^{-j0.2915}= 2.395 e^{j0.494}
\end{align*}
}

\qitem What is $V_{\text{out}}$ in the time domain?

\ws{
\vspace{200px}
}

\sol{
\begin{align*}
% V_{\text{out}}(t)&= \mathfrak{Re}(\widetilde{V}_{\text{out}}e^{j\omega t})
% \intertext{Using Euler's formula, we can say}
% \widetilde{V}_{\text{out}}e^{j\omega t} &= |\widetilde{V}_{\text{out}}|\big(\cos(\omega t+\angle \widetilde{V}_{\text{out}})+j\sin(\omega t+\angle \widetilde{V}_{\text{out}})\big)
% \intertext{Taking the real part, we get}
% \mathfrak{Re}(\widetilde{V}_{\text{out}}e^{j\omega t}) &= |\widetilde{V}_{\text{out}}|\cos(\omega t+\angle \widetilde{V}_{\text{out}}) \\
V_{\text{out}}(t)&= 4.79\cos(1000t+0.494) \text{ V}
\end{align*}
}

\qitem Solve for the current $i(t)$

\ws{
\vspace{200px}
}

    \sol{
    \begin{align*}
    \widetilde{i}&=\frac{\widetilde{V}_{\text{s}}}{Z_R+Z_L+Z_C} = \frac{|\widetilde{V}_{\text{s}}|}{|Z_R+Z_L+Z_C|}e^{j\big(\angle \widetilde{V}_{\text{s}}- \angle (Z_R+Z_L+Z_C)\big)} = 2.395*10^{-4}e^{j2.0647}
    \intertext{Going back to the time domain:}
    i(t)&=4.790*10^{-4}\cos(1000t+2.0647) \text{ A}
    \end{align*}
        }


     \qitem Solve for the transfer function $H(\omega)=\frac{\widetilde{V}_{\text{out}}}{\widetilde{V}_{\text{s}}}$

      Leave your answer in terms of $R$, $L$, $C$, and $\omega$.

    \sol{
    \begin{align*}
    \intertext{Looking back at equation (1)}
    \widetilde{V}_{\text{out}} &= \widetilde{V}_{\text{s}}\frac{Z_C}{Z_R+Z_L+Z_C}
    \intertext{Rearranging we get}
    H(\omega)&=\frac{\widetilde{V}_{\text{out}}}{\widetilde{V}_{\text{s}}}= \frac{Z_C}{Z_R+Z_L+Z_C}=\frac{\frac{1}{j\omega C}}{R+j\omega L+\frac{1}{j\omega C}} \\
    H(\omega) &= \frac{1}{LC(j\omega)^2+jRC\omega+1}
    \end{align*}
    }

\end{enumerate}
