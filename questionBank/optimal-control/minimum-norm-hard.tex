\qns{Weighted Minimum Norm} {\bfseries (Spring 2019 Final)}
\qcontributor{Nikhil Shinde}

\textbf{Minimum Norm Control}
Say we have a controllable system of rank $n$:
\begin{align*}
    \vec{x}(k + 1) = A\vec{x}(k) + \vec{b}u(k)
\end{align*}
 and we want to reach a desired state $\vec{x}_f$ with $k > n$ control inputs. We know we can reach any state in $n$ timesteps, and there are infinitely many ways to reach $\vec{x}_f$ in $k > n$ timesteps. Using the SVD however, we can find the series of control inputs that has the minimum norm. \\
 \newline
 Recall from the section on control that we can use the following linear system to solve for the control inputs:
 \begin{align*}
    \vec{x}(k) = \mathcal{C}_k \vec{u}_k
\end{align*}
\begin{align*}
    \vec{x}(k) = \begin{bmatrix}
        \vec{b} & A\vec{b} & \cdots & A^{k - 1} \vec{b}
    \end{bmatrix} \begin{bmatrix}
        u(k - 1) \\ \vdots \\ u(0)
    \end{bmatrix}
 \end{align*}

To solve for the minimum norm solution, we use the \textbf{Moore-Penrose Pseudoinverse}, a generalization of matrix inverses for rectangular matrices using the full SVD of a matrix. We denote the pseudoinverse with a dagger, and the solution will be $\vec{u}_k = \mathcal{C}_k^{\dagger} \vec{x}(k).$ 
This solution is special because it has \textbf{minimum norm}; that is, if we have another solution $\vec{z},$ such that $\mathcal{C}_k \vec{z} = \vec{x}(k),$ then $\norm{\vec{u}_k} \leq \norm{z}.$

To compute the pseudoinverse, we invert each element of the SVD one at a time:
\begin{align*}
    \mathcal{C}_k \vec{u}_k &= \vec{x}(k) \\
    U \Sigma V^T \vec{u}_k &= \vec{x}(k)\\
    (U^T U) \Sigma V^T \vec{u}_k &= U^T \vec{x}(k) \\
    (\Sigma^{\dagger} \Sigma) V^T \vec{u}_k &= \Sigma^{\dagger} U^T \vec{x}(k) \\
    (V V^T) \vec{u}_k &= V \Sigma^{\dagger} U^T \vec{x}(k) \\
    \vec{u}_k &= V \Sigma^{\dagger} U^T \vec{x}(k) \\
    \implies \vec{u}_k &= \mathcal{C}_k^{\dagger} \vec{x}(k)
\end{align*}

We can calculate $\Sigma^{\dagger}$ accordingly to produce an identity matrix such that the term $\Sigma^{\dagger}\Sigma$ disappears:

$$\Sigma^{\dagger} = \begin{bmatrix} \frac{1}{\sigma_{1}} & 0 &  \cdots & 0 \\ 0 & \frac{1}{\sigma_{2}} & \cdots & 0 \\ \vdots & \vdots & \ddots & \frac{1}{\sigma_{m}} \\ 
    \vdots & \vdots & \ddots & 0 \\ 0 & 0 & 0 & 0 \end{bmatrix}$$

\textit{Note}: This is another, sometimes computationally easier, form of the pseudoinverse:
$$\vec{u}_k = \mathcal{C}_k^T(\mathcal{C}_k\mathcal{C}_k^T)^{-1}\vec{x}(k)$$
See Sp20: Note 11B for more information.

\newpage

You saw in lecture in the context of open-loop control, how we
consider problems in which we have a wide
matrix $A$ and solve $A \vec{x} = \vec{y}$ such that $\vec{x}$
is a minimum norm solution:  
\begin{align*}
\norm{\vec{x}} \leq \norm{\vec{z}} 
\end{align*}
for all $\vec{z}$ such that $A \vec{z} = \vec{y}$. You have worked out how to compute the appropriate ``pseudo-inverse'' for
such wide matrices.

But what if you weren't interested in just the norm of $\vec{x}$? What
if you instead cared about minimizing the norm of a linear transformation $C\vec{x}$? For example, suppose that controls were more or less costly
at different times. 

The problem can be written out mathematically as:

Given a wide matrix  $A$ and a matrix $C$ find $\vec{x}$ such that $A \vec{x} = \vec{y}$ and 
$\norm{C\vec{x}} \leq \norm{C\vec{z}} $ for all 
$\vec{z}$ such that $A \vec{z} = \vec{y}$.


{\bf(10 pts)} Let's start with the case of $C$ being invertible. {\bfseries Solve
  this problem (i.e. find the optimal $\vec{x}$ with the minimum $\|C \vec{x}\|$) for the specific
  matrices and $\vec{y}$ given below. Show your work.}

It is fine to leave your answer as an explicit product of
matrices and vectors.

{\em (HINT: You might want to change variables to solve this
  problem. Don't forget to change back!)} 

\begin{align*}
A = \begin{bmatrix}
1 & 0 & 0 \\
0& 1 & 1
\end{bmatrix}, \quad
C = \begin{bmatrix}
0 & 0 & 2\\
0 & 1 & 0 \\
2 & 0 & 0
\end{bmatrix}, \quad 
\vec{y} = \begin{bmatrix}
2 \\
1
\end{bmatrix}
\end{align*}

For convenience, $C^{-1} = \begin{bmatrix}
0 & 0 & 0.5\\
0 & 1 & 0 \\
0.5 & 0 & 0
\end{bmatrix}$ and you are also given some SVDs on the following page.

\begin{align}
A &= (U_A = \begin{bmatrix}
0& 1  \\
1 & 0
\end{bmatrix})(
\Sigma_A = \begin{bmatrix}
\sqrt{2}  &  0  & 0\\
0 & 1 & 0
\end{bmatrix})(
V^T_A = \begin{bmatrix}
0& \frac{1}{\sqrt{2}}& \frac{1}{\sqrt{2}}  \\
1 & 0 & 0 \\
0& -\frac{1}{\sqrt{2}}& \frac{1}{\sqrt{2}}  
\end{bmatrix}) \\
C &= 
(U_C = \begin{bmatrix}
-1 & 0 & 0  \\
0 & 0 & 1 \\
0 & -1 & 0
\end{bmatrix})
(\Sigma_C = \begin{bmatrix}
2& 0 & 0  \\
0 & 2 & 0 \\
0 & 0 & 1
\end{bmatrix})
(V^T_C = \begin{bmatrix}
0 & 0 & -1  \\
-1 & 0 & 0 \\
0 & 1 & 0
\end{bmatrix}) \\
AC &= 
(U_{AC} = \begin{bmatrix}
0 & 1 \\
1 & 0
\end{bmatrix}
(\Sigma_{AC} = \begin{bmatrix}
\sqrt{5}  &   0 & 0\\
0 & 2 & 0  \\
\end{bmatrix})
(V^T_{AC} = \begin{bmatrix}
\frac{2}{\sqrt{5}}&\frac{1}{\sqrt{5}} & 0 \\
0& 0 & 1 \\
-\frac{1}{\sqrt{5}} & \frac{2}{\sqrt{5}} & 0
\end{bmatrix}) \\
AC^{-1} &= 
(U_{AC^{-1}} = \begin{bmatrix}
0 & 1 \\
1 & 0
\end{bmatrix})
(\Sigma_{AC^{-1}} = \begin{bmatrix}
\frac{\sqrt{5}}{2}  &   0 & 0\\
0 & 0.5 & 0  \\
\end{bmatrix})
(V^T_{AC^{-1}} = \begin{bmatrix}
\frac{1}{\sqrt{5}}&\frac{2}{\sqrt{5}} & 0 \\
0 & 0 & 1 \\
-\frac{2}{\sqrt{5}} & \frac{1}{\sqrt{5}} & 0
\end{bmatrix})
\end{align}

\sol{ 
We already know to find the minimum-norm solution for the system of equations
$$A\vec{x} = \vec{y}$$
using the \textbf{Moore-Penrose Pseudoinverse}. \\

To use that knowledge, we want to first change variables to the C-basis:
$$\widetilde{x} = C\vec{x}$$
Given an arbitrary vector 
$$\vec{p} = C\vec{z}$$
we get the new constraint $\norm{\widetilde{x}}
\leq \norm{\vec{p}}$ for any vector $\vec{p}$ that satisfies
something. What is this something? \\
\newline

Originally, we had $A \vec{x} = \vec{y}$ and, in the new basis, we have $\vec{x} = C^{-1} \widetilde{x}$. 
So, the constraint that needs to be satisfied is 
$$AC^{-1} \widetilde{x} =
\vec{y}$$ 

This is a min-norm problem where the matrix multiplying the vector is $AC^{-1}$. 
To solve this, we find the Moore-Penrose psuedoinverse of $AC^{-1}$:
\begin{align}
\widetilde{x} =
  V_{compact, AC^{-1}} \Sigma_{compact, AC^{-1}}^{-1} U_{AC^{-1}}^T\vec{y}
\end{align}

In this case, we use the compact form of the SVD so the $\Sigma$ matrix is square and therefore invertible.
The compact SVD of $AC^{-1}$ is as follows:
\begin{align}
AC^{-1} &= 
(U = \begin{bmatrix}
0 & 1 \\
1 & 0
\end{bmatrix})
(\Sigma = \begin{bmatrix}
\frac{\sqrt{5}}{2}  &   0 \\
0 & 0.5  
\end{bmatrix})
(V^T = \begin{bmatrix}
\frac{1}{\sqrt{5}}&\frac{2}{\sqrt{5}} & 0 \\
0 & 0 & 1 
\end{bmatrix}).
\end{align}

Calculating this out, we get that $\widetilde{x}
= \begin{bmatrix}\frac{2}{5} \\ \frac{4}{5} \\ 4\end{bmatrix}$. 

However since the original question was to find $\vec{x}$ we have one more substitution to arrive at our final answer:
\begin{align}
\vec{x} &= C^{-1}\widetilde{x} = C^{-1}   V_\Sigma^{-1}U^T\vec{y}
\\
&= \begin{bmatrix}
2 \\
0.8 \\
0.2
\end{bmatrix}.
\end{align}
}